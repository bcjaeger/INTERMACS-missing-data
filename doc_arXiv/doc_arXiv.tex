\documentclass{article}

\usepackage{arxiv}

\usepackage[utf8]{inputenc} % allow utf-8 input
\usepackage[T1]{fontenc}    % use 8-bit T1 fonts
\usepackage{hyperref}       % hyperlinks
\usepackage{url}            % simple URL typesetting
\usepackage{booktabs}       % professional-quality tables
\usepackage{amsfonts}       % blackboard math symbols
\usepackage{nicefrac}       % compact symbols for 1/2, etc.
\usepackage{microtype}      % microtypography
\usepackage{lipsum}

\title{Improving Outcome Predictions for Patients Receiving Mechanical
Circulatory Support by Optimizing Imputation of Missing Values}

\author{
    Byron C. Jaeger
    \thanks{Source code available at
\url{https://github.com/bcjaeger/INTERMACS-missing-data}}
   \\
    Department of Biostatistics \\
    University of Alabama at Birmingham \\
  Birmingham, AL 35211 \\
  \texttt{\href{mailto:bcjaeger@uab.edu}{\nolinkurl{bcjaeger@uab.edu}}} \\
  }


\begin{document}
\maketitle

\def\tightlist{}


\begin{abstract}
\textbf{Background} Risk predictions play an important role in clinical
decision making. When developing risk prediction models, practitioners
often impute missing values to the mean. The purpose of this article is
to evaluate the impact of applying different strategies to impute
missing values on the prognostic accuracy of prediction models fitted to
the imputed data. A secondary objective was to compare the accuracy of
different imputation methods. To complete these objectives, we used data
from the Interagency Registry for Mechanically Assisted Circulatory
Support (INTERMACS). \newline\textbf{Methods and Results} We applied all
pairwise combinations of different strategies to impute missing values
and three different strategies to fit a risk prediction model for
mortality and transplant after receiving mechanical circulatory support.
Model performance was compared using Concordance (i.e.~C-index), Brier
Score, and Net reclassification index. Results indicated that multiple
imputation consistently resulted in superior prognostic models compared
to other missing data strategies, particularly imputation to the mean.
\newline\textbf{Conclusion} Selecting an optimal strategy to handle
missing values can have a substantial impact on improving model
accuracy. In the current analysis, multiple imputation emerged as an
optimal strategy to handle missing values in the INTERMACS data. The
evaluation and selection of the optimal missing data strategy in this
work has the potential to improve risk predictions for other
longitudinal registries.
\end{abstract}

\keywords{
    Missing Data,
   \and
    INTERMACS,
   \and
    imputation,
   \and
    heart failure,
   \and
    mortality,
   \and
    risk prediction
  }

\hypertarget{introduction}{%
\section{Introduction}\label{introduction}}

\label{sec:introduction}

Heart disease is a leading cause of death in the United States. Heart
failure, a primary component of heart disease, affects over 6 million
Americans, and for \textasciitilde10\% of these patients medical
management is no longer effective
\cite{benjamin2017heart,national2017health}. Mechanical circulatory
support (MCS) is a surgical intervention in which a mechanical device is
implanted in parallel to the heart to improve circulation
\cite{patel2014contemporary}. Typically, MCS is used while a patient
waits for a heart transplant (bridge-to-transplant) or in some cases as
an alternative to transplant (destination therapy)
\cite{slaughter2009advanced}. Over 250,000 patients could benefit from
MCS \cite{miller2011left}. However, less than 4,000 new patients receive
a long-term MCS device each year, with widely heterogeneous outcomes
\cite{stewart2011keeping}. The 2-year survival on MCS ranges from 61\%
for destination therapy to 78\% for bridge-to-transplant
\cite{patel2014contemporary}. Therefore, there is great need for
reliable predictions of patient-specific probability to experience
adverse events after receiving MCS. This information can be used to
improve patient selection for MCS,inform the design of next generation
pumps, and refine patient management strategies.

\bibliographystyle{unsrt}
\bibliography{references.bib}


\end{document}
